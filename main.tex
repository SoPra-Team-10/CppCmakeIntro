\documentclass[aspectratio=169]{beamer}

\usepackage[utf8]{inputenc}
\usepackage{soul}
\usepackage{pdfpcnotes}

\usetheme{default}
\usecolortheme{dove}

\title{C++, CMake und GTest}
\author{Paul Nykiel}
\date{\today}

\begin{document}
\maketitle
\frame{\tableofcontents[currentsection]}

\section{Einleitung}
\begin{frame}
    \frametitle{Was ist C++}
    \begin{itemize}
        \item \st{C with classes} \pnote{Ursprünglicher Name}
        \item C++11! \pnote{Neuer Standard inzwischen 17 bald 20}
        \item Standardisiert und offen \pnote{Viele Compiler}
    \end{itemize}
\end{frame}

\begin{frame}
    \frametitle{C++ im Vergleich zu Java, C\#}
    \begin{itemize}
        \item Undefiniertes Verhalten 
            \pnote{Null-Pointer, fehlendes Return, division durch Null. Kann nicht kontrolliert werden z.B. Microcontroller}
        \item Keine automatische Speicherverwaltung 
            \pnote{Performanter Code! Systemnah! Wenn benötigt liefert die STL passende Container, nicht der Compiler}
        \item Kleiner Sprachkern 
            \pnote{60 Schlüsselwörter, C\# z.B. 86. Keine UI, kein Networking...}
        \item Templates 
            \pnote{Nicht generics, viel mächtiger, z.B. Boost MSM, sichere Pointer (Guideline support library)}
        \item Operatorenüberladung
            \pnote{Bsp. Vektor}
        \item Tendentiell weniger tiefe Vererbung
            \pnote{Z.b. Container und Iteratoren, keine Überklasse aber Eigenschaften (forward-Iterator), dadurch weniger Boilerplate}
        \item Mehrfachvererbung
    \end{itemize}
\end{frame}

\section{Buildsystem}

\end{document}
